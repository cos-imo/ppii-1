\chapter*{Applications Numériques Existantes}

Depuis quelques années, consommateurs comme agriculteurs prennent peu à peu conscience des coûts, dangers et pollutions inhérents au système de consommation habituel: coûts en eau, en transports, coûts liés aux nombreux intermédiaires...
C'est pourquoi de nombreuses instances liées à la production/distribution de denrées (AMAP, circuits courts etc...) virent le jour. Cette partie s'attachera donc à l'étude de ceux-ci.

\section*{Les AMAP}
Les AMAP - Association pour le Maintien d'une Agriculture Paysanne - sont des associations se proposant de mettre en relation agriculteurs et consommateurs. L'intérêt est ainsi de réduire le circuit de distribution des denrées, et de bénéficier des nombreux avantages en résultant: réduction des coûts liés aux intermédiaires (créant ainsi à la fois une prix plus attractifs pour le consommateur et un bénéfice plus grand pour l'agriculteur), bénéfices pour l'environnement grâce à la suppression de tout ou partie du transport, de la dépense d'énergie liée à la conservation des aliments etc...\\ \\
La création et le fonctionnement d'une AMAP reposent sur la motivation et le bon vouloir de tout-un-chacun. En effet, selon le site officiel des AMAPs:\\
"Une AMAP naît en général de la rencontre d'un groupe de consommateurs et de paysans (ou artisans transformateurs) prêts à entrer dans la démarche."\footnote{http://www.reseau-amap.org/}\\

Conformément à l'idée initiale du projet, le principe des AMAPs est de lutter contre le gaspillage alimentaire; ainsi, toujours selon le site officiel des AMAPs, section "Qu'est ce qu'une AMAP": \\
"Contrairement à la grande distribution, les consommateurs en AMAP accordent moins d'importance à la standardisation des aliments ; tout ce qui est produit est consommé (alors que dans l'autre cas, ce peut être jusqu'à 60 \% de la récolte qui reste au champ). Ce principe est d'une part très valorisant pour le paysan, et d'autre part il permet de diminuer le prix des denrées en reportant les coûts sur la totalité de la production."

