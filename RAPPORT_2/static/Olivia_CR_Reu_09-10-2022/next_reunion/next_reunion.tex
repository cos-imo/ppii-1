\section{Prochaine réunion}
    \noindent
    \textit{\underline{Date prévue:}} 
    
    \vskip 0.1 cm
    
    % FORMAT : Jour DD mois AAAA. Ex: Mercredi 14 octobre 2022
    % CA PEUT TRÈS BIEN ÊTRE "non défini"
    Lundi 10 Octobre 2022
    
    \vskip 0.25cm

    \noindent
    \textit{\underline{Lieu:}} 
    
    \vskip 0.1 cm
    % CA PEUT TRÈS BIEN ÊTRE "non défini"
    Discord, en ligne
    
    \vskip 0.25cm
    
    \noindent
    \textit{\underline{Ordre du jour (non exhaustif) :}}
    \begin{itemize}[label=\textbullet]
        \item REPRISE DES ELEMENTS À TRAITER : voir l'avancement des tâches
    \end{itemize}
    
    \vskip 0.25cm
    
    \noindent
    \textit{\underline{TO-DO list}} :

    \vskip 0.1 cm
    
    \noindent
    \begin{tabularx}{\textwidth}{|p{3.2cm}|X|}
        % A MODIFIER
        \hline
        \centerline{Léo VESSE} 
        & Contacter l'AMAP qui est en lien avec Marché \\
        
        \hline
        
        \centerline{Olivia AING}
        & Faire la liste des contacts des AMAP / producteurs\\
        \cline{2-2}
        & Faire la page de couverture de l’État de l’Art\\
        
        \hline
        
        \centerline{Cosimo UNGARO}
        & Se renseigner sur ce qui existe : applications, sites, associations\\
        \cline{2-2}
        & Appeler les contacts pour savoir :
            \begin{itemize}[label=\textbullet]
                \item ce qu'ils utilisent
                \item ce qui les intéresserait (comme fonctionnalités)
                \item ce dont ils auraient besoin
            \end{itemize}\\
        
        \hline
        
        \centerline{Thomas LERUEZ}
        & Se renseigner sur ce qui existe : applications, sites, associations\\
       
        \hline
    \end{tabularx}
    
    \vskip 2.0cm
    
    \begin{center}
        \textbf{FIN DE LA RÉUNION}
    \end{center}