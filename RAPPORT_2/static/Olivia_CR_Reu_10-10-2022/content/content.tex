\section{Échanges au cours de la réunion}

    % -----
    % PRÉSENTATION EN DESCRIPTION
    % -----

        Une fois tout le monde présent et les problèmes de son (plus ou moins) réglés, la réunion débute.

        Cosimo nous fait passer le \href{https://docs.google.com/document/d/1kN3yLbHacPv0A-VE_MOV0zmUfvvmW8FzXHhSFT4Q7lM/edit?usp=sharing}{premier brouillon de l’état de l’art}.

        \vskip 0.5cm
    
    \begin{description}

        \item [Se renseigner sur les plateformes existantes] \hfill\\
            Cosimo et Thomas se mettent d’accord pour chercher des informations sur le plus de sites de circuits courts sur \href{https://www.economie.gouv.fr/dgccrf/Publications/Vie-pratique/Fiches-pratiques/produits-alimentaires-commercialises-en-circuits-courts-0}{la liste donnée par Cosimo}.
            \par
            Cosimo a aussi fouillé le site officiel des AMAP, et s’est penché sur les paiements : il n’y a pas de site à proprement parler, les producteurs doivent se débrouiller avec des e-mails et des brochures. Pour les paiements, les chèques sont conseillés pour laisser une trace.
            \par
            Cosimo loue Leaf. Léo ajoute que ce serait bien d’en plus de récupérer les fonctionnalités, faire des recherches un peu plus en profondeur.
            \par
            Cosimo propose de faire la plateforme de paiement plus tard. Olivia propose de passer par PayPal, soutenue par Thomas.

            
        \vskip 0.5cm
        
        \item[Nom de l’équipe et du site] \hfill\\
            Nous nous mettons d’accord sur le fait que ce n’est pas nécessaire dans l’immédiat. Cela aura lieu à un brainstorming ultérieurement.

        \vskip 0.5cm
        
        \item[Écriture des Conditions Générales d’Utilisation] \hfill\\
            	Ce n’est pas nécessaire dans l'immédiat non plus. Cosimo propose de faire cela plus tard, et Olivia à la fin. Cosimo propose d’utiliser un générateur (semblable à celui pour le dropshipping), et Léo propose de le développer nous-même. Olivia propose de demander de l’aide à Mme Heurtel lorsque le temps viendra. Elle rajoute qu’il faudra noter au fur et à mesure du projet les points qu’il faudra aborder dans les CGU.
                \par
                Thomas rajoute qu’il faudra faire les RGPD (tout ce qui concerne les droits d’exploitation et de conservation des données personnelles).
                Léo propose de mettre tout ça en place sous la forme d’un arbre ou d’une mind map.

        \vskip 0.5cm
                
        \item[Lecture du document brouillon de l’état de l’art] \hfill\\
                Le document a été validé, la construction et le contenu convient à l’ensemble de l’équipe (Merci Cosimo !)

        \vskip 0.5cm

        \item[Consultation de l’enchaînement des pages web] \hfill\\
            La première version est disponible sur l’état de l’art.

            \begin{itemize}[label=\textbullet]
                \item Page d’accueil (Cosimo)\\
                    Olivia propose de mettre le total de CO2 et le total d’argent qui a été reversé aux producteurs sur la page d’accueil, pour montrer aux utilisateurs le résultat de leurs efforts, en chiffres.
                    \par
                    Pour la carte, celle de Google Maps est payante, donc il faudra en trouver une autre.
                \item Dashboard client (Cosimo)\\
                    Thomas propose de mettre une bande avec les grammes de CO2 économisé.
                    \par
                    Cosimo ajoute qu’il faudrait mettre l’argent économisé en ne passant pas par des distributeurs.
                    \par
                    Léo complète avec le montant que l’utilisateur a reversé aux producteurs.
                \item Dashboard producteur (Cosimo)\\
                    Rien ajouté.
                
                \item Page d’informations (Léo)\\
                    Léo propose de rajouter une page d’information pour pouvoir nous contacter en cas de souci. On créerait alors également un support ou un feedback à travers les coordonnées que l’on mettra pour le site.
                    \par
                    Thomas ajoute qu’il faudrait faire un ticket côté client et côté producteur pour reporter les éventuels problèmes.
                
                \item Sécurité\\
                    On impose une sécurisation par double authentification (via numéro de téléphone).
                    \par
                    Ainsi, en cas de souci, on peut blacklister le numéro de téléphone / l’adresse IP / l’adresse MAC…
                    \par
                    Thomas dit que ça ne sera probablement pas possible, mais Olivia répond qu’au moins, on en aura parlé.
                    
                \item Autre\\
                Design
                \par
                Olivia est chargée de faire les icones de fruits et légumes
                \par
                Logistique
                \par
                Le financement du projet se ferait par dons, ainsi, tout l’argent récolté peut être restitué aux producteurs.
                Il faudra créer une adresse mail officielle pour recevoir les feedbacks.
                \par
                L’idée de départ, interprétée par Olivia, était de faire une plateforme de producteur à particulier dans un premier temps, pour garantir les transactions.



            \end{itemize}

        

    
    \end{description}
