\chapter*{Applications Numériques Existantes}

Depuis quelques années, consommateurs comme agriculteurs prennent peu à peu conscience des coûts, dangers et pollutions inhérents au système de consommation habituel: coûts en eau, en transports, coûts liés aux nombreux intermédiaires...
C'est pourquoi de nombreuses instances liées à la production/distribution de denrées (AMAP, circuits courts etc...) virent le jour. Cette partie s'attachera donc à l'étude de ceux-ci.



\section*{Circuits-courts}

Source(s): \begin{itemize}
	\item $[1]$ Produits alimentaires commercialisés en circuits courts | economie.gouv.fr
	\item $[2]$ Les circuits courts, un levier important pour s’approvisionner en produits frais | Ministère de l'Agriculture et de la Souveraineté alimentaire
	\item $[3]$ Frais et local
	\item $[4]$
	\item $[5]$ Manger local à Lyon
	\item $[6]$ La Ruche qui dit Oui !
	\item $[7]$ Bienvenue à la ferme
	\item $[8]$ Le Marché Vert
	\item $[9]$ Solidarite-occitanie-alimentation.fr
	\item $[10]$ Dinan Producteurs confinés
	\end{itemize}

La vente de produits alimentaires en circuits courts, qui, d’après la définition officielle, concerne les produits alimentaires vendus via au plus un intermédiaire
Les produits alimentaires commercialisés en vente directe, c’est-à-dire les produits vendus directement du producteurs au consommateur (remise en mains propres, aucun intermédiaires)
Les points de vente collectifs / magasins de producteurs, regroupant des producteurs venus vendre leur production aux consommateurs; le critère retenu étant le fait qu’au moins 70% du chiffre d’affaire doit provenir de ce type de vente
Les AMAP, Associations pour le maintien d’une agriculture paysanne.

La principale source d’informations liées aux circuits courts semble être la page du ministère officiel de l’agriculture. Sans proposer de plateforme à l’échelle nationale, celle-ci renvoie vers divers sites indépendants:
	$[3]$ Frais et Local
	$[4]$ Le Marché Vert
	$[5]$ Manger Local Lyon
Le principal intérêt du présent projet doit donc porter sur une application simple à prendre en main et fonctionnant sur une échelle nationale, afin de pouvoir centraliser la revente de fruits et légumes de particuliers à particulier sur une seule et unique plateforme.


\section*{LEAF}
	Source: [1] LEAF
LEAF, pour Localiser Et Acheter Frais est une application de revente de produits frais montée par une particulière sur la région de Toulouse. Concentrée sur le “surplus de potager”, cette appli permet à tout un chacun de s’inscrire afin de revendre au prix souhaité les productions de son jardin.
Centrée sur la revente de fruits et légumes, cette application est donc au coeur de notre problématique.

Autres sites et applications

	Source(s):
	\begin{itemize}
	De particulier à particulier
	\item $[1]$ Particuliers
	\item $[2]$ Lepotiron.fr
	\item $[3]$ Proposez vos surplus de fruits et légumes ! - Le Potager d'à Côté
	\item $[4]$ Fruit and Food
	\item $[5]$ Fresh
De professionnel à particulier (vente directe producteur)
\item $[6]$ Pourdebon
\item $[7]$ Acheter à la source
\item $[3]$ Manger français, direct producteurs et transformateurs
\item $[4]$ Vente directe producteur
\item $[5]$ Chez Vos Producteurs
\item $[6]$ Direct Potager
\item $[7]$ Vente directe à la ferme ou sur les marchés des produits du terroir, produits frais et fermiers, Magasins de producteurs, supermarchés à la ferme - Bienvenue à la ferme
\end{itemize}


\section*{Les AMAP}
Les AMAP - Association pour le Maintien d'une Agriculture Paysanne - sont des associations se proposant de mettre en relation agriculteurs et consommateurs. L'intérêt est ainsi de réduire le circuit de distribution des denrées, et de bénéficier des nombreux avantages en résultant: réduction des coûts liés aux intermédiaires (créant ainsi à la fois une prix plus attractifs pour le consommateur et un bénéfice plus grand pour l'agriculteur), bénéfices pour l'environnement grâce à la suppression de tout ou partie du transport, de la dépense d'énergie liée à la conservation des aliments etc...\\ \\
La création et le fonctionnement d'une AMAP reposent sur la motivation et le bon vouloir de tout-un-chacun. En effet, selon le site officiel des AMAPs:\\
"Une AMAP naît en général de la rencontre d'un groupe de consommateurs et de paysans (ou artisans transformateurs) prêts à entrer dans la démarche."\footnote{http://www.reseau-amap.org/}\\

Conformément à l'idée initiale du projet, le principe des AMAPs est de lutter contre le gaspillage alimentaire; ainsi, toujours selon le site officiel des AMAPs, section "Qu'est ce qu'une AMAP": \\
"Contrairement à la grande distribution, les consommateurs en AMAP accordent moins d'importance à la standardisation des aliments ; tout ce qui est produit est consommé (alors que dans l'autre cas, ce peut être jusqu'à 60 \% de la récolte qui reste au champ). Ce principe est d'une part très valorisant pour le paysan, et d'autre part il permet de diminuer le prix des denrées en reportant les coûts sur la totalité de la production."


