Motivation et objectifs principaux du présent document
Ce document est un état de l'art des systèmes de redistribution de fruits et légumes entre particuliers et professionnels. Ainsi, nous tenterons au travers de ce document d'étudier:
\begin{itemize}
\item{Les systèmes de redistribution de professionnel à particulier avec peu (ou pas) d'intermédiaires, appellés "circuits courts"} 
\item{Le(s) système(s) de redistribution de particuliers à particuliers}
\end{itemize}
Pour cela, une étude à été menée sur Internet, sur divers sites web que vous pourrez retrouver dans la bibliographie en fin de document. Nous tentons ici de connaître les différentes infrastructures existants déjà afin de comprendre comment celles-ci fonctionnent, quelles améliorations pourraient y être apportées et quelles fonctionnalitées nouvelles pourraient être implémentées afin de faciliter un tel transfert de consommables. Comme cité précedemment, l'étude se fondera majoritairement sur deux axes:

\begin{itemize}
	\item Tout d'abord l'étude des systèmes existants de "circuits courts", constitants en groupes de produc
teurs ayant décidés, dans un but de lutter contre la "malbouffe" de se constituer en associations pour revendre leurs produits directement aux consommateurs. Le but affiché est de réduire le nombre d'intermédiaires afin de non seulement promouvoir les productions locales auprès des consommateurs, mais également de pouvoir en réduire le prix (le nombre d'intermédiaires moins élevé incluant naturellement une baisse des coûts usuellement liée à la marge que prennent ces derniers) pour les consommateurs et ainsi rendre les produits locaux plus accessibles, mais également de pouvoir augmenter de la même façon la rémunération des agriculteurs. 
	\item Mais également la redistribution de produits de particulier à particulier. En effet, l'idée de ce projet émane de M. Olivier Festor (Référent Projet et Algorithmie) qui constate une perte d'une grande partie des fruits produits dans son jardin. Souhaitant lutter contre ce gâchis inutile, il suggère alors l'idée d'une application Web permettant aux particuliers d'éviter ce genre de perte. Nous nous attacherons donc également dans ce document aux difficultées sous-jacente à ce pan du projet: comment permettre la sécurité de chacun lors de l'utilisation d'une telle application? Il va de soi qu'il n'est pas raisonnable d'attendre d'un utilisateur qu'il rentre son adresse et laisse des inconnus s'introduire dans sa propriété afin de ramasser des fruits.
\end{itemize}
