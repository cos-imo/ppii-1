\chapter*{Fonctionnalités à implémenter}
\textbf{La page d'accueil} devra rester simple, afin d’être accessible pour un public peu porté sur l’informatique. Dans cet esprit de synthétisme, nous proposons donc d’afficher d’emblée les fonctionnalités suivantes sur celle-ci:
\begin{itemize}
	\item[-] Carte suggérant les producteurs / marchés les plus proches
	\item[-] Une courte description du projet
	\item[-] Liste des prochaines réunions
	\item[-] Des boutons Plus d’informations / Se connecter / S’inscrire
\end{itemize}

\textbf{L’interface utilisateur} prendra la forme d’un Dashboard, sur lequel seront regroupées toutes les fonctionnalités de base afin de permettre à l’utilisateur moyen de prendre facilement et rapidement en main la plateforme, et de gérer tout son trafic depuis une page unique. Ainsi elle comprendra:
\begin{itemize}
	\item[-] Une messagerie instantanée
	\item[-] Une liste des nouvelles les plus récentes
	\item[-] Une liste des prochains évènements
	\item[-] Une carte des produits
	\item[-] Une première interface de commande, renvoyant vers un applet plus complet
\end{itemize}

\textbf{L’interface producteur} sera également un DashBoard, lui permettant de:
\begin{itemize}
	\item[-] Une messagerie instantanée
	\item[-] Une plateforme pour renseigner ses productions
	\item[-] Un système de consultation de ses commandes
	\item[-] Un moyen de consulter ses revenus
\end{itemize}

\textbf{La page d'accueil} comprendra
\begin{itemize}
	\item[-] Les dernières nouvelles des producteurs locaux
	\item[-] Une Carte des producteurs les plus proches basés sur votre géolocalisation (et si la loc est bloquée bah on met Nancy par chauvinisme)
	\item[-] Une liste des prochaines réunions par ordre chronologique
\end{itemize}

\textbf{La page ‘Infos’} comprendra
\begin{itemize}
	\item[-] Des infos sur les producteurs locaux
	\item[-] Des infos sur les AMAP et les prochaines réunions
\end{itemize}

Bien entendu de nombreuses autres fonctionnalités non citées ici pourront être implémentées; parmi celles-ci figurent:
\begin{description}
   \item[Un algoritme pour proposer un panier] Afin de réduire le temps/la distance à parcourir par l'utilisateur.
   \item[Un algorithme du plus court chemin] pour proposer à l'utilisateur un itinéraire optimal pour aller récupérer ses produits 
   \item[Un système de feed] Pour proposer à l'utilisateur les informations les plus intéressantes selon son profil (utilisation de cookies?)
   \item[Des propositions de recettes] Selon les fruits et légumes de saison, ceux dans son panier, ceux disponibles chez les producteurs les plus proches...
\end{description}
